\title{\huge 物理学の数学的方法}
\author{ohataken}
\maketitle

\newpage

% ================================

\section{微分形式}

ある物理の教科書を読んでいると突然「ここでは通常の微分ではなく、微分形式を用いる」と書かれていた。著者は理論物理学者で、その著書はどれも数学的なスタイルで書かれているから、やはりこれも高度な数学を使うのだなぁと思って、よくわからずに本を閉じてしまった。微分形式という名前だけは印象に残った。それからしばらくして、 Twitter で数学、物理界隈を多くフォローするようになると、彼らが微分形式の話をしているのを見た。微分形式は素晴らしい道具だ、これで物理を書き直すべきだと熱く語っていた。

だんだんと気になってきてしまったので、まず Google で検索した。ぜひ同じくそうしてみてほしい。おそらく可微分多様体上の共変テンソル場だとか、接ベクトル場とか余接ベクトル場とか、よくわからない言葉で説明されてしまうと思う。なお、東京大学理学部数学科のシラバスを調べてみると、学部 3 年生が選択履修する講義として位置付けられている。検索してすぐわかるような簡単なものではなさそうだ。だがあきらめずに横着していきたい。

いろいろと見聞きしたことをわかる範囲でまとめると、微分形式とは全微分のようなもので、そのメリットは座標変換に対しても形が変わらないことらしい。

ベクトル解析を使ったことがあるならば、微分形式はその強化版と説明することもできる。たとえば、私は大学の課題でラプラシアンを直交座標から極座標に書き換えたことがあったが、あれは大変だった。微分形式を使うと、かなり楽に済んでしまう。加えて、ベクトル解析の外積は3次元と7次元の場合にしか成立しないらしい。物理学者はともかく数学者はとても気に入らないだろう。

\hrulefill

\paragraph{ウェッジ積}

これから$ dx \wedge dy $みたいな式がたくさん出てくる。この山のような見た目の演算子をウェッジ積とよぶ。

\paragraph{微分1形式}

微分1形式は、こんな見た目をしている。ここまではふつうの全微分みたいだ。

\begin{equation*}
\begin{split}
f dx + g dy + h dz
\end{split}
\end{equation*}

微分1形式は曲線のほんのひと区画、線素をあらわす。

\paragraph{微分2形式}

微分2形式は、こんな見た目だ。

\begin{equation*}
\begin{split}
f dx \wedge dy + g dy \wedge dz + h dz \wedge dx
\end{split}
\end{equation*}

山のような形の中置演算子はウェッジ積という。ウェッジ積の演算はあとで書く。微分2形式は曲面のごく小さい区画、面積素をあらわす。

\paragraph{微分3形式}

微分3形式は、こんな見た目だ。

\begin{equation*}
\begin{split}
f dx \wedge dy \wedge dz
\end{split}
\end{equation*}

微分3形式は、体積のごく小さい区画、体積素をあらわす。

\paragraph{微分n形式}

微分n形式は、ここまでのことから類推したとおりのものだ。

\begin{equation*}
\begin{split}
f dx_1 \wedge dx_2 \wedge ... \wedge dx_n
\end{split}
\end{equation*}

ウェッジ積が中置演算子のせいで、和の記法のシグマ記号や、積の記法のパイ記号のようにきれいには書けない。ウェッジ積を中置の2項演算子ではなく、多変数関数のようにみたてて、こうやって書くこともある。

\begin{equation*}
\begin{split}
f \wedge_{i=1}^{n} dx_i
\end{split}
\end{equation*}

\paragraph{ウェッジ積の交代性}

$ d x \wedge d y = - d y \wedge d x $

\paragraph{ウェッジ積の分配法則}

$ d x \wedge ( d y + d z ) = d x \wedge d y + d x \wedge d z $

\paragraph{ウェッジ積の結合法則}

$ d x \wedge (d y \wedge d z) = (d x \wedge d y) \wedge d z $

\paragraph{外微分}

関数を外微分すると、微分1形式になる。微分1形式を外微分すると、微分2形式になる。微分$n$形式を外微分すると微分$n+1$形式になる。

関数$f(x,y)$があったとき、微分1形式は

\begin{equation*}
\begin{split}
d \wedge f \\
&= (\frac{\partial}{\partial x} dx + \frac{\partial}{\partial y} dy) \wedge f\\
&= \frac{\partial f}{\partial x} dx + \frac{\partial f}{\partial y} dy
\end{split}
\end{equation*}

となる。ここまでは全微分のように見える。この微分1形式をふたたび外微分すると、

\begin{equation*}
\begin{split}
&d \wedge df \\
&= d \wedge \frac{\partial f}{\partial x} dx + d \wedge \frac{\partial f}{\partial y} dy \\
&= (\frac{\partial}{\partial x} dx + \frac{\partial}{\partial y} dy) \wedge \frac{\partial f}{\partial x} dx \\
& \; \; \; + (\frac{\partial}{\partial x} dx + \frac{\partial}{\partial y} dy) \wedge \frac{\partial f}{\partial y} dy \\
&= \frac{\partial}{\partial x} dx \wedge \frac{\partial f}{\partial x} dx 
+ \frac{\partial}{\partial y} dy \wedge \frac{\partial f}{\partial x} dx \\
& \; \; \; + \frac{\partial}{\partial x} dx \wedge \frac{\partial f}{\partial y} dy 
+ \frac{\partial}{\partial y} dy \wedge \frac{\partial f}{\partial y} dy \\
\end{split}
\end{equation*}

このあと紹介する交換法則をつかって計算を進めて、

\begin{equation*}
\begin{split}
&= 0 + \frac{\partial}{\partial y} \frac{\partial f}{\partial x} dy \wedge dx \\
& \; \; \; + \frac{\partial}{\partial x} \frac{\partial f}{\partial y} dx \wedge dy + 0 \\
&= - \frac{\partial}{\partial y} \frac{\partial f}{\partial x} dx \wedge dy + \frac{\partial}{\partial x} \frac{\partial f}{\partial y} dx \wedge dy \\
&= ( \frac{\partial}{\partial x} \frac{\partial f}{\partial y} - \frac{\partial}{\partial y} \frac{\partial f}{\partial x} ) dx \wedge dy
\end{split}
\end{equation*}

\paragraph{引き戻し}

微分形式を変換する? 操作。あとで書く。

\paragraph{ホッジ作用素}

微分1形式は線素を、微分2形式は面積素をあらわす。微分1形式にホッジ作用素を作用させると、線素からそれと垂直な面積素を得る。微分2形式にホッジ作用素を作用させると、面積素からそれと垂直な線素を得る。いまは3次元空間の話をしたが、一般の$n$次元空間では微分$m$形式にホッジ作用素を作用させると微分$n-m$形式を得る。あとで書く。

\paragraph{内部積}

微分形式の次数下げ。あとで書く。

% ================================

\section{線形形式}

数学の概念の名前なんて、どうせ発見者に由来するものばかりで、名前から意味を推し量ろうなんてナンセンスだと、わかっているのだけれども、微分形式という名前は気になってしまう。微分はわかっているから、形式の方はなんなのだ、微分じゃない形式はあるのか、たとえば積分形式はあるのか。

線形形式とは写像のひとつで、線形性というよい性質を備え、ベクトルや行列をうけとって、スカラーをひとつ返すものだ。引数がひとつのものを線形1形式、ふたつのものを線形2形式、$n$個のものを線形$n$形式という。行列式は線形1形式の例、ベクトルの内積は線形2形式の例となっている。

微分形式はおそらくこの線形形式から名前をもらっている。

\newpage

% ================================

\section{多重線形写像}

おおざっぱに言って、線形を「よい」と読み替えても通ると思う。線形なものはよいもので、線形代数はよい代数だ。非線形なものはよくない。多重線形写像にいたっては、もう幾重にもよさが合わさったすばらしい写像なのだろう。

ただの線形写像とは$n=1$の場合で、それが$n$回重なるので$n$次線形写像と呼ばれる。

\hrulefill

まず線形写像について。以下のふたつの条件を満たすとき、$ f $を線形写像という。

\paragraph{条件 1 加法}

$ f(x + y) = f(x) + f(y) $

\paragraph{条件 2 スカラー倍}

$ f(ax) = a f(x) $

この写像$f$を$n$次に拡張したものを多重線形写像という。$n=2$のときは特別に双線形写像という。

\newpage

% ================================

\section{双対}

微分形式のよさとは、座標変換に対して不変ということだった。なぜそんな都合のいいことが起こるかというと、微分形式は互いに表と裏の関係があるふたつのものでできていて、座標変換されてもその変化が打ち消されるという仕組みによるらしい。この都合のいい互いに表と裏の関係のことを、数学では双対(そうつい)というようだ。

\newpage

% ================================

\section{可微分多様体}

可微分多様体というものを研究するなかで、微分形式は発見された。可微分多様体とは微分することができるが、直交座標なのか極座標なのか、具体的にどういう座標系にあるものなのかはわからない、というもので、このことによって座標系がなんなのか明言せずに、それでも空間についてなにか言うことができる。可微分多様体がそういうものなので、座標に依存しないで微分を計算したいというのは自然な動機だとわかる。そのために双対とかテンソル積とか、そういう性質をそなえたものを代数から取り込んだのだと思う。

\newpage

% ================================

\section{位相空間}

位相空間は数学者のための数学だと思う。数学を使いたいだけの私にとって、位相空間は原理的すぎる。こんな浅はかな態度だから理解できないのかもしれないが。

数学者にとっては、位相空間を考えることで「近い」ということを数学の言葉にすることができて、全単射よりすこし強い同相写像ということを定義したり、距離とはどんな性質を備えるべきか定義することができるらしい。そういえば図書館や書店で「集合と位相」という本の背表紙を見たことがある。集合をやってから位相をやるのが教科書的ということだろう。

\newpage

\newpage

