\title{\huge 物理学の数学的方法}
\author{ohataken}
\maketitle

\newpage

% ================================

\section{微分形式}

ある物理の教科書を読んでいると「ここでは通常の微分ではなく、微分形式を用いる」と堂々と書かれていた。著者は理論物理学者で、その著書はかなり数学的なスタイルで書かれているから、やはりこれも高度な数学を使うのだなぁと思って、微分形式だけが印象に残って、それだけで本を閉じてしまった。それからしばらくして、 Twitter で数学、物理界隈を多くフォローするようになると、彼らが微分形式の話をしているのを見た。微分形式は素晴らしい道具だ、これで物理を書き直すべきだと熱く語っていた。だんだんと気になってきてしまった。

まずは Web を検索、と思い微分形式を Google 検索した、ぜひ同じくそうしてみてほしい。おそらく可微分多様体上の共変テンソル場だとか、接ベクトル場とか余接ベクトル場とか、もっとよくわからない言葉で説明されてしまうと思う。なお、東京大学理学部数学科のシラバスを調べてみると、学部 3 年生が選択履修する講義として位置付けられている。検索してすぐわかるような簡単なものではなさそうだ。だがあきらめずに横着していきたい。

いろいろと見聞きしたことをわかる範囲でまとめると、微分形式とは全微分のようなもので、そのメリットは座標変換に対しても形が変わらないことらしい。

ベクトル解析を使ったことがあるならば、微分形式はその強化版と説明することもできる。たとえば、私は大学の課題でラプラシアンを直交座標から極座標に書き換えたことがあったが、あれは大変だった。微分形式を使うと、かなり楽に済んでしまう。加えて、ベクトル解析の外積は3次元と7次元の場合にしか成立しないらしい。物理学者はともかく数学者はとても気に入らないだろう。

\hrulefill

微分形式とは、まずは全微分だと思っておくことにする。ただし以下の計算ルールに従う必要はある。

\paragraph{交代性}

$ d x \wedge d y = - d y \wedge d x $

\paragraph{分配法則}

$ d x \wedge ( d y + d z ) = d x \wedge d y + d x \wedge d z $

\paragraph{結合法則}

$ d x \wedge (d y \wedge d z) = (d x \wedge d y) \wedge d z $

\paragraph{外微分}

微分形式にはそれぞれ次数があって、微分1形式とか微分2形式とか呼ばれている。外微分は微分1形式から微分2形式を得る操作で、つまり微分$n$形式から微分$n+1$形式を得ることができる。あとで書く。

\paragraph{ホッジ作用素}

微分1形式は線素を、微分2形式は面積素をあらわす。微分1形式にホッジ作用素を作用させると、線素からそれと垂直な面積素を得る。微分2形式にホッジ作用素を作用させると、面積素からそれと垂直な線素を得る。いまは3次元空間の話をしたが、一般の$n$次元空間では微分$m$形式にホッジ作用素を作用させると微分$n-m$形式を得る。あとで書く。

\paragraph{内部積}

微分形式の次数下げ。あとで書く。

% ================================

\section{線形形式}

数学の概念の名前なんて、どうせ発見者に由来するのだから、名前から意味を推し量ろうなんてナンセンスだと、わかっているのだけれども、微分形式という名前はとにかく気になってしまう。

線形代数の分野に、線形形式という概念がある。微分形式はこの線形形式に由来しているようだ。

\newpage

% ================================

\section{多重線形性}

おおざっぱに言って、線形を「よい」と読み替えても通ると思う。線形なものはよいもので、線形代数はよい代数だ。非線形なものはよくない。多重線形性にいたっては、もう幾重にもよさが合わさったすばらしいものなのだろう。

\newpage

\newpage

